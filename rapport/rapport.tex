\documentclass{article}
\usepackage[utf8]{inputenc}
\usepackage[T1]{fontenc}
\usepackage[french]{babel}
\usepackage[parfill]{parskip}
\usepackage{amsmath}
\usepackage{amssymb}
\usepackage{amsfonts}
\usepackage{graphicx}
\usepackage{subfigure}
\usepackage[font={small}]{caption}
\usepackage{float}
\usepackage{listingsutf8}
\usepackage{fullpage}
\usepackage[nochapter]{vhistory}
\usepackage{hyperref}
\usepackage{titlesec}
\usepackage{xcolor}
\usepackage{verbatim}
\usepackage{graphicx}
\usepackage{subcaption}
\usepackage{comment}
\usepackage{adjustbox}


\newcommand*{\MyIncludeGraphicsMaxSize}[2][]{%
\begin{adjustbox}{max size={\textwidth}{\textheight}}
    \includegraphics[#1]{#2}%
\end{adjustbox}
}

% -----------------------------------------------------
% -----------------------------------------------------
% -----------------------------------------------------

\hypersetup{
%couleurs des liens cliquable changée pour une meilleur lisibilité
    colorlinks=true,
    linkcolor=blue,
    filecolor=magenta,
    urlcolor=cyan,
    pdfpagemode=FullScreen,
    }


\title{Degenerescence - Coloration - Coeurs}
\author{Robin Becker, Noé Bourgeois }
\date{March 2022}

\begin{document}

\maketitle
\tableofcontents
\newpage

\section{Introduction}

\newpage

\section{Degénérescence}
\subsection{Introduction}
\subsection{algorithme}
\subsubsection{Description}
\subsubsection{Complexité}
\subsubsection{Mise en oeuvre en Java }

\newpage

\section{Coloration}
\subsection{Introduction}
\subsection{algorithme}
\subsubsection{Description}
\subsubsection{Complexité}
\subsubsection{Mise en oeuvre en Java }

\newpage

\section{Degénérescence et Cœurs }
\subsection{Introduction}
\subsection{algorithme}
\subsubsection{Description}
\subsubsection{Complexité}
\subsubsection{Mise en oeuvre en Java }

\newpage

\section{Experiences sur des graphes issus de données réelles }

\subsection{Ressources }
\href{https://snap.stanford.edu/data/.}{https://snap.stanford.edu/data/.}

\subsection{Graphes }
\begin{itemize}
    \item 1
    \begin{itemize}
        \item dégénérescence
        \item colorabilité
        \item cœurs
        \end{itemize}

    \item 2
    \begin{itemize}
        \item dégénérescence
        \item colorabilité
        \item cœurs
    \end{itemize}
\end{itemize}

\newpage

\section{Ressources}
\underlined{(cf. énoncé)}

Redaction scientifique:

\href{http://informatique.umons.ac.be/algo/redacSci.pdf.}{http://informatique.umons.ac.be/algo/redacSci.pdf.
}

Ressources bibliographiques:

\href{https://www.bibtex.com/.}{https://www.bibtex.com/.}


Classes de la bibliothèque Java
algs4.jar, disponible à l’adresse suivante :

\href{https://algs4.cs.princeton.edu/code/.}{https://algs4.cs.princeton.edu/code/.}

\end{document}
